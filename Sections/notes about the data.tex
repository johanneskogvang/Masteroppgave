\chapter{Notes about the data}

\section{MLE's unlimited}

For an example where one person chooses after one box, see line 44, ID 'kanhil'. Big alpha, mles look good. 

Why does it make sense that the $\alpha$'s are 0 when the participants does not open all boxes? There are two reasons to make a decision before you really have to:
\begin{enumerate}
    \item $\alpha > 0$ 
    \item The participant is not really following what the are doing, making decisions arbitrarily. Then, $\eta$ will typically be small, close to zero, and then $\alpha$ could be zero.
\end{enumerate}



For the discusssion: maybe the limited trial where the test terminated after 3 boxes impacted the decisions in the subsequent limited trials? Although it might not look like that as so many of the parameters are zero. 

Person 58 acts optimally (chooses when there are 6 boxes of one of the colours) for unlimited. Thus, for this parametric bootstrapping, the probabilities are either 1 or 0, and the simulations we end up doing are all the same as what the persn actually does. then we get the same estimates for alpha and eta for all of teh simulations, and the CI's are really just one number. One can therefore argue that this prametric bootstrapping is not the best choice for this person. 


\section{Abbreviations}
\begin{itemize}
    \item JTC = Jumping to conclusions
    \item MLE = Maximum Likelihood Estimator
    \item IO = Ideal observer
    \item CI = Confidence Intervals
\end{itemize}


Bør finne IO solution for hver av de 9 trialsene? eller for noen av de og sammenlikne med noen perosner, og dermed stille parameterene jeg har funnet i lys av dette. eller sammenlikne parameterene med hvor nærme de er IO solution. Deet går jop ikke når IO er avh v parametererne???