\newpage
\chapter{Background Theory}
%\section{Theory}
In this chapter we will go through some of the statistical theory used in this report. That includes the theorem of total probability, Bayes' theorem, the beta and gamma functions, Bayesian modelling, loss functions, the softmax function, maximum likelihood estimation and bootstrapping.  

\section{The Theorem of Total Probability}
The theorem of total probability is often used when we want to find some probability, and this probability is hard to find. Then, sometimes it might be easier to find that probability if we condition on something, and use the theorem of total probability. 
\begin{theorem}[Theorem of Total Probability, Continuous Variables]
If we have a continuous variable, $\Theta$, and a discrete variable, $U$, and both $P(U=u|\Theta=\theta)$ and  $f_\Theta(\theta)$ are known for all $\theta$, then we can find $P(U=u)$ from \citep{schay2016introduction} 
\begin{equation}
    \label{lawoftotprob}
    P(U=u) = \int_{-\infty}^{\infty} P(U=u|\Theta=\theta)f_{\Theta}(\Theta=\theta) \: \dd \theta.
\end{equation}
\end{theorem}

\begin{comment}
In \citet{schay2016introduction}, the theorem of total probability for continuous variables is stated as 
\begin{theorem}[Theorem of Total Probability, Continuous Versions]
 For a continuous random variable Y and any event A, if $f_{Y|A}$ and $f_Y$ exists for all y, then
\begin{equation}
   % \label{lawoftotprob}
    P(A) = \int_{-\infty}^{\infty}
    P(A|Y=y)f_Y(y) dy.
\end{equation}
\end{theorem}
\end{comment}




Consider, for example, two discrete random variables $U$ and $V$ that are conditionally independent given the continuous stochastic variable $\Theta$. To find the probability that $U+V$ is equal to some integer $j$, we can use the theorem of total probability to condition on theta. Thus,
\begin{equation*}
    P(U+V=j) = \int_{-\infty}^\infty P(U+V=j|\Theta=\theta)f_{\Theta}(\Theta=\theta) \: d\theta.
\end{equation*}
Later, we can exploit the conditional independence. If $\theta$ is a probability that is defined on the interval (0,1), this will be integrated on that interval, such that 
\begin{equation*}
    P(U+V=j) = \int_{0}^1 P(U+V=j|\Theta=\theta)f_{\Theta}(\Theta=\theta) \: d\theta.
\end{equation*}
%Mer utfyllende her?




\section{Bayes' Rule}
Bayes' rule can be used to find conditional probabilities and distributions. 
\begin{theorem}[Bayes' Rule]
Consider two events, $A$ and $B$. We can find the probability of $A$ given event $B$ by the use of the probability of event $B$ given $A$ and the probabilities of the events $A$ and $B$ separately \citep{statinf}. Hence,
\begin{equation}
\label{bayesrule}
    P(A|B)=\frac{P(B|A)P(A)}{P(B)}.
\end{equation}
\end{theorem}




\begin{comment}
From \citet{schay2016introduction}, we have a version of Bayes' theorem for continuous variables which is stated as 
\begin{theorem}[Bayes' Theorem]
\label{bayestheorem}
For a continuous random variable Y and any event A with nonzero probability, if $P(A|Y=y)$ and $f_Y$ exist for all $y$, then
\begin{equation}
    \label{bayestheorem_eq}
    f_{Y|A}(y) = \frac{P(A|Y=y)f_Y(y)}{\int_{-\infty}^{\infty}P(A|Y=y) f_Y(y) dy}.
\end{equation}
Here $f_Y$is called the prior density of $Y$, and $f_{Y|A}$ its posterior density, referring to the fact that these are the densities of $Y$ before and after the observation of $A$. 
\end{theorem}

From \eqref{lawoftotprob} we see that the denominator is the probability of event $A$, \begin{equation*}
    \int_{-\infty}^{\infty}P(A|Y=y) f_Y(y) = P(A),
\end{equation*}
such that Bayes' theorem can be stated as
\begin{equation}
    \label{bayestheorem_eq2}
    f_{Y|A}(y) = \frac{P(A|Y=y)f_Y(y)}{P(A)}.
\end{equation}


Or:
\end{comment}

\begin{comment}
\begin{theorem}[Bayes' Theorem]
\label{bayestheorem}
Consider two continuous random variables, $U$ and $\Theta$, that both have a nonzero probability. If both $P(U=u|\Theta=\theta)$ and $P(\Theta=\theta)$ exist for all $\theta$, then 
\begin{equation}
    \label{bayesrule}
    P(\Theta=\theta|U=u) = \frac{P(U=u|\Theta=\theta)P(\Theta=\theta)}{\int_{-\infty}^{\infty} P(U=u|\Theta=\theta)P(\Theta=\theta) \: \dd \theta}.
\end{equation}
$P(\Theta=\theta)$ is the prior probability/density of $\Theta$, and it represents the prior knowledge we have about that parameter. $P(\Theta=\theta|U=u)$ is called the posterior of $\Theta$, which is the probability density after we have observed that $U=u$ \citep{schay2016introduction}. 
\end{theorem}


From \eqref{lawoftotprob} we see that the denominator is the probability that $U=u$,
\begin{equation*}
    \int_{-\infty}^{\infty} P(U=u|\Theta=\theta)P(\Theta=\theta) \: \dd \theta = P(U=u).
\end{equation*}
Thus, Bayes' theorem can be reformulated as
\begin{equation}
    \label{Bayesrule2}
     P(\Theta=\theta|U=u) = \frac{P(U=u|\Theta=\theta)P(\Theta=\theta)}{P(U=u)}.
\end{equation}


As an example, consider a random variable $U$ that is binomial distributed with parameters $n$ and $\theta$. Thus,
\begin{equation*}
    U|\theta \sim \mathrm{Binomial}(n,\theta).
\end{equation*}
Using Theorem \ref{bayestheorem} and \eqref{Bayesrule2}, we get that the posterior probability of $\Theta|U$ is
\begin{equation*}
    P(\Theta=\theta|U=u) = \frac{P(U=u|\Theta=\theta)P(\Theta=\theta)}{P(U=u)}.
\end{equation*}

\end{comment}





As an example, consider a discrete random variable, $U$. We can find the probability that $U$ is greater than or equal to 7, and condition on it being different from six by using \eqref{bayesrule}. Then $U \geq 7$ is an event, and $U\neq6$ is another event. Thus,
\begin{equation}
    P(U\geq 7|U\neq6) = \frac{P(U\neq6|U\geq7)P(U\geq7)}{P(U\neq6)}.
\end{equation}



\section{The Beta and Gamma Functions}
%The beta distribution is continuous on the interval between 0 and 1, and have two parameters \citep{statinf}. Consider a parameter $\theta$ is beta distributed with parameters $\gamma$ and $\kappa$, hence,
%\begin{equation*}
%    \theta \sim beta(\gamma,\kappa).
%\end{equation*}
%The probability density function of $\theta$ is then
% \begin{equation}
    % \label{betadistribution}
    % f(\theta|\gamma,\kappa) = \frac{1}{\mathrm{B}(\gamma,\kappa)} \theta^{\gamma-1}(1-\theta)^{\kappa-1},\:\: 0<\theta<1, \gamma>0, \kappa>0,
% \end{equation}
% where $\mathrm{B}(\gamma,\kappa)$ is the beta function, meaning that


Later, we will use the beta and gamma functions and some properties of these. These are therefore stated here. This theory can for example be found in \citet{statinf}. The gamma function for a parameter $\kappa$ is 
\begin{equation*}
    \label{gamma_func}
    \Gamma(\kappa) = \int_0^\infty t^{\kappa-1}e^{-t} \dd t.
\end{equation*}
A useful property of the gamma function is that it is recursive. Hence,
\begin{equation}
\label{gamma_recursive_property}
    \Gamma(\kappa+1) = \kappa \Gamma(\kappa), \quad \kappa>0 .
\end{equation}

Additionally, the beta function with parameters $\gamma$ and $\kappa$ is defined as
\begin{equation*}
    \mathrm{B}(\gamma,\kappa) = \int_0^1 \theta^{\gamma-1}(1-\theta)^{\kappa-1} \: \dd \theta.
\end{equation*}
We can express the beta function as a product of gamma functions. That yields
\begin{equation}
\label{beta_as_gamma}
    \mathrm{B}(\gamma,\kappa) = \frac{\Gamma(\gamma)\Gamma(\kappa)}{\Gamma(\gamma+\kappa)}.
\end{equation}



\section{Bayesian Modelling}
\label{theory_bayesian_modelling}
Consider a stochastic variable, $U$, that has a probability density function $f(u|\theta)$, where $\theta$ is a parameter upon which $U$ depends. In classical statistics, $\theta$ is said to be a fixed but unknown value. On the other hand, in Bayesian statistics, we consider $\theta$ as a stochastic variable. Thus, $\theta$ has a density function, $f(\theta)$. This is called the prior distribution as it represents the prior knowledge we have about $\theta$ before observing any data. That could be our own subjective believes about the parameter, or based on other previously collected data or studies. One could also choose a prior distribution that does not say anything about the parameter at all. This is called a non-informative prior, and it is often used when we have none or little prior information about the parameter \citep{givens2012computational}. 
If we have collected data, we can update our prior beliefs with the information we get from that data. The resulting distribution is called the posterior distribution of $\theta$. This can be found using Bayes' theorem, and it includes both the prior information we have and the new information we get from the data. 

Consider a discrete stochastic variable, $U$, that has a sampling distribution $P(U=u|\theta)$, and let $P(U=u)$ be the marginal distribution of $U$. Additionally, let $f(\theta)$ be the prior distribution of $\theta$, hence our prior beliefs of the parameter. Using Bayes' rule as it is stated in \eqref{bayesrule}, we get that the posterior distribution of $\theta$ given $u$, $f(\theta|u)$, can be expressed as \citep{statinf}
\begin{equation*}
    f(\theta|u) = \frac{P(U=u|\theta)f(\theta)}{P(U=u)}.
\end{equation*}
We can sometimes exploit the fact that the posterior distribution is proportional to the numerator in the above expression. This is because the denominator is a normalising constant. Then,
\begin{equation}
    \label{posterior_proportional}
    f(\theta|u) \propto P(U=u|\theta)f(\theta).
\end{equation}
If \eqref{posterior_proportional} have the form of a known distribution, then that known distribution is the posterior distribution. 

As an example, consider a random variable, $U$, that is binomially distributed with parameters 12 and some probability, $\theta$. Thus,
\begin{equation*}
    U|\Theta=\theta \sim \mathrm{Binomial}(12,\theta).
\end{equation*}
Hence, the probability that we have $u$ successes out of twelve, given $\theta$, is
\begin{equation}
\label{binomial_12_ex}
    f(u|\theta) = \binom{12}{u} \theta^{u} (1-\theta)^{12-u}.
\end{equation}
As $\theta$ is a probability, its value is on the interval $[0,1]$. We know that the beta distribution is conjugate with the binomial distribution and has value between 0 and 1 \citep{statinf}, thus we choose a beta prior for $\theta$ with parameters $\gamma$ and $\kappa$. Hence,
\begin{equation}
\label{theta_with_beta_prior}
    \Theta \sim \mathrm{Beta}(\gamma,\kappa). 
\end{equation}
The prior density of $\Theta$ is then
\begin{equation}
    \label{betadistribution}
    f(\theta) = \frac{1}{\mathrm{B}(\gamma,\kappa)}\theta^{\gamma-1}(1-\theta)^{\kappa-1}.
\end{equation}
We can find the posterior distribution of $\theta$ using \eqref{posterior_proportional}, \eqref{binomial_12_ex} and \eqref{betadistribution}. Thus,
\begin{equation*}
    \begin{aligned}
        f(\theta|u) 
        &\propto f(u|\theta)f(\theta)\\[6pt]
        &\propto \binom{12}{u} \theta^{u} (1-\theta)^{12-u} \frac{1}{\mathrm{B}(\gamma,\kappa)}\theta^{\gamma-1}(1-\theta)^{\kappa-1}
    \end{aligned}
\end{equation*}
All the factors that do not include $\theta$ are constants, and we collect them together as one constant, denoted $C$. Then
\begin{equation*}
    \begin{aligned}
        f(\theta|u) 
        \propto C \: \theta^{u+\gamma-1}(1-\theta)^{12-u+\kappa-1}.
    \end{aligned}
\end{equation*}
We can see that this is proportional to a beta distribution like the one in \eqref{betadistribution}, but in this case with parameters $u+\gamma$ and $12-u+\kappa$. Hence, the posterior distribution is a beta distribution with these parameters, 
\begin{equation*}
    \Theta|U=u \sim \mathrm{Beta}(u+\gamma,12-u+\kappa).
\end{equation*}






\section{Loss Functions}
\label{theory_loss_functions}
A loss function typically says something about the cost, or loss, of an action related to a parameter. Let $\Omega_{\delta}$ be the action space, consisting of all the actions that can be done, where $\delta$ is an action. Then,
\begin{equation*}
    \delta \in \Omega_{\delta}.
\end{equation*}
Additionally, let $z$ be the true, but unknown state of nature, where 
\begin{equation*}
    z \in \Omega_z.
\end{equation*} 
We can define a loss function that depends on $z$ and $\delta$, which we denote $L(z,\delta)$. This is then the loss when making a decisions, $\delta$, regarding $z$ \citep{statisticalDecisionTheoryLiese2008}.

A loss function could for example be the 0-1-loss function. If for example $\Omega_{\delta}= \Omega_z = \{0,1\}$, the loss function could be
\begin{equation}
\label{loss_func_indicator}
    L(z,\delta) = I(z \neq \delta),
\end{equation}
where $I$ is an indicator function such that
\begin{equation*}
    L(z,\delta) =
    \begin{cases}
        0,&  \text{if } z = \delta, \\
        1,&  \text{if } z \neq \delta.
    \end{cases}
\end{equation*}


\begin{comment}

se statistical decison theory for hva en loss func er. på s 109 står det om 0-1-loss. 

If the decision you make is far away from the parameter, the cost of making that decision will be high, and then we say that the loss will be big. A decision close to the parameter will give a small loss \citep{statinf}. The loss function gives the loss of making a decision. 

A commonly used loss function is the absolute error loss function. Let $\delta$ be the 
action represented as a real number, and let $\theta$ be the parameter we make decisions regarding. The absolute error loss function is then defined as
\begin{equation*}
    L(\theta,\delta) = |\theta-\delta|.
\end{equation*}
The loss is zero if $\delta=\theta$. A special case of this loss function is the 0-1-loss. Then the loss is either 0 or 1, and only two decisions can be made. Then, 
\begin{equation}
\label{loss_func_indicator}
    L(\theta,\delta) = I(\theta \neq \delta),
\end{equation}
where $I$ is an indicator function such that
\begin{equation*}
    L(\theta,\delta) =
    \begin{cases}
        0,&  \text{if } \theta = \delta, \\
        1,&  \text{if } \theta \neq \delta.
    \end{cases}
\end{equation*}
\end{comment}

In some cases, we would like to find the expected value of the loss function. Taking the expected value of an indicator function gives the probability that the event is happening \citep{algdat}. Hence, taking the expectation of \eqref{loss_func_indicator} gives
\begin{equation}
\label{expectation_of_loss_func_general}
    E[L(z,\delta)] = E[I(z\neq\delta)] = P(z\neq\delta).
\end{equation}



As an example, consider the box task with twelve boxes that could be either blue or red once they are opened. We define a stochastic variable, $X_i$, that represents the colour of the $i$-th opened box, such that 
\begin{equation*}
    X_i =
    \begin{cases}
        0,& \text{if box }i \text{ is blue,}\\
        1,& \text{if box }i \text{ is red.}
    \end{cases}
\end{equation*}
When $i$ boxes are opened, let $\textbf{X}_{1:i}$ denote the colours of the i boxes, such that
\begin{equation*}
    \textbf{X}_{1:i} = (X_1,X_2,...,X_{i}).
\end{equation*}
Additionally, let $Z$ be the colour that is in majority when all twelve boxes are opened, the true majority colour. This is also as stochastic variable as it depends on the colours of the twelve boxes, the $X_i$'s. We define $Z$ as
\begin{equation}
\label{def_of_Z}
    Z = I\left(\sum_{j=1}^{12}X_j > 6\right).
\end{equation}
Then,
\begin{equation}
\label{Z_true_majority}
    Z = 
    \begin{cases}
        0,& \text{if blue is the true majority colour,} \\
        1,& \text{if red is the true majority colour.}
    \end{cases}
\end{equation}
Also, let $\delta$ be the choice the participant makes about which colour that is the dominant colour, such that
\begin{equation*}
    \delta = 
    \begin{cases}
        0,& \text{if the participant chooses blue as the majority colour,}\\
        1,& \text{if the participant chooses red as the majority colour}.
    \end{cases}
\end{equation*}
We can then define a loss function for the choice that the participant makes. This can be a 0-1 loss as in \eqref{loss_func_indicator}, and the loss function can therefore be defined as 
\begin{equation}
\label{loss_func_example}
    L(Z,\delta) = I(Z \neq \delta),
\end{equation}
Then, the loss is 0 if the participant chooses the right colour as the majority colour, and 1 if the wrong colour is chosen. 

To find the expected loss, we take the expectation of the loss function. As $Z$ depends on the colours of the twelve boxes, we have to condition on the colour of the boxes that already are opened, $ X_{1:i}=x_{1:i}$. The expectation of the loss function is then
\begin{equation*}
    E[L(Z,\delta)|X_{1:i}=x_{1:i}] = E[I(Z\neq \delta)|X_{1:i}=x_{1:i}].
\end{equation*}
As in \eqref{expectation_of_loss_func_general}, this expectation is the probability that $\delta \neq Z$, but here the probability is dependent on $X$. Thus,
\begin{equation}
\label{exp_loss_theory}
    E[L(Z,\delta)|X_{1:i}=x_{1:i}] = P(Z\neq \delta|X_{1:i}=x_{1:i}).
\end{equation}



%(Not sure if I should include this as this was fine when the loss func was if red or blue was teh right color, but now I have 3 decisions, and not two, and then teh 0-1-loss does not count, but it counts if we only talk about the two first decisions, decision 0 and 1.)





\section{The Law of Total Expectation}
Let $\{A_1,A_2,...,A_k\}$ be a partition of the sample space, $S$. Thus, there are $k$ non-overlapping parts, such that $A_i \cap A_j = \emptyset$, $\forall \:
i \neq j$. Then we also have that $S = A_1 \cup A_2 \cup...\cup A_k$. If we want to find the expectation of an event, $B$, and we have the expectation of $B$ on each of these partitions, we can use the law of total expectation. It states that
\begin{equation}
    E[B] = \sum_i E[B|A_i]P(A_i).
\end{equation}
This can also be used to find the expectation of functions \citep{schay2016introduction}. Let $g(B)$ be the function that we want to take the expectation of, then
\begin{equation}
\label{law_tot_exp_func}
    E[g(B)] = \sum_i E[g(B)|A_i]P(A_i).
\end{equation}

Later we will use the law of total expectation when we find the expectation of a loss function that says something about the loss of opening the next box in the box task. This expected loss is dependent on the colour of the box that will be opened. Thus, to find that expected loss, we use the law of total expectation and condition on the colour of the next box. 

%Again consider the box task. Let $\delta_i=2$ denote the choice of opening another box when $i$ boxes already are opened. We can in addition to the loss functions found above, \eqref{loss_func_example}, find a loss function for the choice of opening the next box. This loss function is dependent on the choice that are made after the next box is opened as well. We denote these as $\delta_{(i+1):12}$, and thus, we denote this loss function as $L(Z,\delta_i=2,\delta_{(i+1):12})$ (senere kaller jeg disse videre valgene for $IO(\textbf{x}_{i},0)$, men det blir litt mange greier å ta med her. hva skal jeg gjøre md det?). 

%The expectation of this loss function is dependent on the colour of the next box, $X_{i+1}$. We do not know the colour of that box, but we can find probabilities for the box being blue and red given the colour of the first $i$ boxes, $P(X_{i+1}=0|\textbf{x}_i)$ and $P(X_{i+1}=1|\textbf{x}_i)$, respectively. We can then find the expectation of this loss function using \eqref{law_tot_exp_func}, and condition on $X_{i+1}$. Then, 
%\begin{equation*}
%    E[L(Z,\delta_i=2,\delta_{(i+1):12})|\textbf{x}_i] = \sum_{j=0}^1
%    E[L[Z,\delta_{(i+1):12}|\textbf{x}_i]P(X_{i+1}=j|\textbf{x}_i).
%\end{equation*}
%Er dette litt mye greier å ta med her? Feks mtp notasjonen. Og er notasjonen godt nok forklart her?






\section{The Softmax Function} \label{section_theory_softmax}
The softmax function is commonly used in classification problems with more that two classes \citep{softmax}. Consider a decision, $\Delta$, which now is a stochastic variable for which we want to construct a distribution. We find a probability mass function for $\Delta=\delta$ using a softmax function. Let there be $D$ decisions, such that
\begin{equation*}
    \delta \in \{0,1,2,...,D-1\}.
\end{equation*}
Additionally, let $\EE_{\delta}(\vp)$ be values tied to each decision that depends on some parameters, $\vp$. The probability mass function for each decision, $\delta$, could be found using a softmax function, such that
%\begin{equation*}
%    \begin{aligned}
%        f(\delta_i|\textbf{x}) = \frac{\text{exp}(- \eta \EE_{\delta_i})}{\sum_{j=0}^{K-1} \text{exp}(-\eta \EE_{\delta_j})}.
%    \end{aligned}
%\end{equation*}
%Er et riktig å ha $|\textbf{x}$ her? jeg vil jo ha $|\alpha,\beta,\eta$ i min modell? burfe jeg ha $|\eta$, også senere introdusere $\alpha$ og $\beta$. slik at det heller blir som under?
\begin{equation}
\label{softmax_in_theory}
    \begin{aligned}
        f(\delta|\vp,\eta) = \frac{\text{exp}(- \eta \EE_{\delta}(\vp))}{\sum_{d=0}^{D-1} \text{exp}(-\eta \EE_{d}(\vp))},
    \end{aligned}
\end{equation}
where $\eta$ is some parameter. 


These decisions could for example be the three choices you have each time a box is opened in the box task. These choices are that blue is the majority colour, or that red is, denoted $\delta=0$ and $\delta=1$, respectively. The last choice is to open another box, which is denoted $\delta=2$. Then, we let $\EE_0(\vp)$ be the expected loss when choosing that blue is the majority colour, similarly to \eqref{exp_loss_theory}. Additionally, we let $\EE_1(\vp)$ and $\EE_2(\vp)$ be the expected loss of choosing red as the majority colour and of opening another box, respectively. Then, the probability mass function for $\delta$ could be as in \eqref{softmax_in_theory}. We then have a probability mass function for each of the three decisions that depends on the expected losses, parameters $\vp$ and some parameter $\eta$.



\section{Maximum Likelihood Estimation}
\label{section_theory_mle}

Maximum likelihood estimation is used to find estimates for parameters in a distribution. These are the estimates that, as the name implies, maximises the likelihood, and for short, we call them MLEs. Assume that we have probability distribution for a stochastic variable, $\Delta$, and consider $n$ samples, $\delta_1,\delta_2,...,\delta_n$, of $\Delta$. Denote the probability mass function for each of these $\delta$'s as $f(\delta|\vp)$, where $\vp$ contains the parameters in the probability mass function. If the $\delta_i$'s are independent, the likelihood function is defined as
\begin{equation}
\label{likelihood}
    L(\vp|\delta_1,\delta_2,...,\delta_n) =  \prod_{i=1}^{n} f(\delta_i|\vp).
\end{equation} 
The MLEs are then the estimates of $\vp$ that maximises this function, and they are often denoted $\hat{\vp}$. It is often hard to maximize the likelihood function, then it might be easier to take the logarithm of the likelihood function and maximize that instead. This is called the log likelihood function, and is normally denoted as $l$. Thus,
\begin{equation}
\label{chap2:log_likelihood}
    \begin{aligned}
        l(\vp|\delta_1,\delta_2,...,\delta_n) 
        =& \text{log}\left(L(\vp|\delta_1,\delta_2,...,\delta_n)\right)\\
        =& \text{log}\left(\prod_{i=1}^{n} f(\delta_i|\vp) \right).
    \end{aligned}
\end{equation}
As the logarithm of products is the sum of the logarithms, we get that the log likelihood is
\begin{equation}
\label{chap2:log_likelihood2}
    \begin{aligned}
        l(\vp|\delta_1,\delta_2,...,\delta_n) = \sum_{i=1}^n \text{log}(f(\delta_i|\vp)).
    \end{aligned}
\end{equation}
Maximizing this will give the same maximum point as if we maximize the likelihood function \citep{statinf}. 

As an example, consider that the $\delta_i$'s have probability mass function as in \eqref{softmax_in_theory}. The parameters that we want to find estimates for are then $\vp$ and $\eta$. If we have $n$ samples of $\Delta$, denoted $\delta_i$, where $i \in \{1,2,...,n\}$,
the likelihood function would be
\begin{equation*}
    \begin{aligned}
        L(\vp,\eta|\delta_1,\delta_2,...,\delta_n) 
        =& \prod_{i=1}^{n} f(\delta_i|\vp,\eta)\\[6pt]
        =& \prod_{i=1}^{n}
        \frac{\text{exp}(- \eta \EE_{\delta_i}(\vp))}{\sum_{d=0}^{2} \text{exp}(-\eta \EE_{d}(\vp))}.
    \end{aligned}
\end{equation*}

The log likelihood would then be
\begin{equation*}
    \begin{aligned}
        l(\vp,\eta|\delta_1,\delta_2,...,\delta_n) =& \sum_{i=0}^N \text{log}\left( \frac{e^{-\eta \EE_{\delta_{i}}(\vp)}}
        {\sum_{d=0}^{2} \text{exp}(-\eta \EE_{d}(\vp))}\right) \\[6pt]
        =& \sum_{i=0}^N \left(
        -\eta \EE_{\delta_{i}} 
        - \text{log} \left( \sum_{d=0}^{2} \text{exp} \left(-\eta \EE_{d}(\vp)\right) \right) \right).
    \end{aligned}
\end{equation*}
The maximum likelihood estimators of $\vp$ and $\eta$ would then be the values that maximises this log likelihood function. We denote them as $\hat{\vp}$ and $\hat{\eta}$.




\section{Bootstrapping}
\label{section_theory_bootstrap}
Consider a sample, $(\delta_1,\delta_2,...,\delta_n)$, where the samples are identically and independently distributed from an unknown distribution, $F$. We can use this sample to estimate this distribution, thus, find $\hat{F}$. To get some ideas about the properties of $F$, we can find the properties of $\hat{F}$. Sometimes it is challenging to do this analytically, and we can then use simulations instead. This is where bootstrapping is useful. Bootstrapping is a way of finding those samples, either from the original sample, $(\delta_1,\delta_2,...,\delta_n)$, or from the estimated distribution, $\hat{F}$. We can then use those samples to for example find standard error, bias, variance, or perhaps the most common, confidence intervals \citep{bootstrap}.


%a sample (iid) from an unknown distribution ($F$). can use that sample to estimate the distribution, hence find $\hat{F}$. To get some knowledge about the properties of $F$, we can find out things about the properties of $\hat{F}$. Often hard to do this analytically, thus, instead we use simulations. Then we can use bootstrapping. 


%Bootstrapping is a way of doing statistical inference using samples from a dataset. It can be used to measure the accuracy of parameters. We can for example find standard errors, bias, variance, or perhaps the most common, confidence intervals. In that way we do not need any formulas, only many samples \citep{bootstrap}. However, with many samples, we are dependent on a computer. 

There are two types of bootstrapping, nonparametric and parametric. In the nonparametric bootstrap, $\hat{F}$ is the empirical distribution of the data, and we take samples from our original sample. Consider for example that you have a dataset $\boldsymbol{\delta}=(\delta_1,\delta_2,\delta_3,\delta_4,\delta_{5})$. A bootstrap sample of this might then be $(\delta_5,\delta_5,\delta_2,\delta_3,\delta_1)$ and another might be $(\delta_2,\delta_{4},\delta_{2},\delta_{2},\delta_{1})$. These are resampled versions of $\boldsymbol{\delta}$. Thus, the bootstrap samples consists of elements from the original dataset, but some of them might not appear at all in a bootstrap sample while others might appear more than once. Drawing $B$ of these samples, we can do inference about the population the original data is from. 

In the parametric bootstrap, we make assumptions about the population, and $\hat{F}$ is the parametric distribution. Consider a sample, ($\delta_1,\delta_2,...,\delta_n$), from a distribution that has a probability mass function $f(\delta|\vp)$, where $\vp$ might be a vector of parameters \citep{statinf}. We can for example find an estimate for $\vp$, $\hat{\vp}$, using maximum likelihood estimation as in Chapter \ref{section_theory_mle}. When we have done that, we can draw new samples, denoted $\delta_i^*$ from $f(\delta|\hat{\vp})$, such that
\begin{equation*}
    \delta_1^*,\delta_2^*,...,\delta_n^* \sim f(\delta|\hat{\vp}).
\end{equation*}
If we draw $B$ samples, we can, as for the nonparametric bootstrap, do inference. 



%Bootstrapping is a way to draw or simulate many samples from one single dataset. If you have a dataset, you can draw random samples from them with replacement, to construct bootstrap samples \citep{bootstrap}. If you for example have data $\textbf{x}=(x_1,x_2,x_3,x_4,x_{5})$, then a bootstrap sample might be $(x_5,x_5,x_2,x_3,x_1)$ and another might be $(x_2,x_{4},x_{2},x_{2},x_{1})$. These are then resampled versions of $\textbf{x}$. Thus, the bootstrap samples consists of elements from the original dataset, but some of them might not appear at all in a bootstrap sample while others might appear more than once. This is called nonparametric bootstrapping. If we, for example, have found the maximum likelihood estimate (MLE) of a parameter, $\eta$, we can use these bootstrap samples to for example find the standard error or confidence interval (CI) for $\eta$.

%If we have a distribution for the $x$'s, we could instead of using a nonparametric bootstrapping, use a parametric version. Then we simulate new $x$'s based on the MLE of $\eta$. If the distribution is $f(x|\eta)$, and the MLE of $\eta$ is denoted $\hat{\eta}$, then we simulate new $x$'s with $f(x|\hat{\eta})$, and get new samples, $(\hat{x}_1,\hat{x}_2,\hat{x}_3,\hat{x}_4,\hat{x}_5)$. As for the nonparametric bootstrap, we can simulate many samples, and for example find standard errors and confidence intervals (CIs) for parameters. 

\subsection{Confidence Intervals with Bootstrap Samples}
\label{theory_ci_bootstrap}
One way of doing inference is to find confidence intervals. When we have $B$ bootstrap samples, there are multiple methods for finding these.  
A confidence interval (CI) for a parameter is an interval that will contain the true value of the parameter a given proportion of the times an interval is constructed. If we for example have a 90\% CI, then the true value of the parameter will be in the interval 90\% of the times an interval is constructed \citep{bootstrap}.

One method for finding CIs with bootstrap samples is the percentile method. The percentile method is simple to both understand and implement. However, these confidence intervals might be biased. Then, one could instead use approaches such as \textit{bias corrected and accelerated} intervals or \textit{approximate bootstrap confidence} intervals. For simplicity, we are in this report using the percentile confidence intervals. 


Consider a situation with $B$ bootstrap samples. Let the vector $\vp$ be one parameter. If the intention is to find a CI for $\vp$, we find the MLE of each of the $B$ samples and plot those samples in a histogram. If we want to find a 90\% confidence interval using the percentile method, then we find the 5-th and 95-th percentiles. The 5-th percentile is the value of $\hat{\vp}$ in the histogram where 5\% of the samples are below. The 95-th is where 5\% of the values are above. This is visualised in Figure \ref{percentile_ci_example}. Here we have 150 bootstrap samples, and the MLE of $\vp$ is found for each sample. These values are plotted in a histogram, where the red dashed lines represent the 5 and 95 percentiles. That means that 5\% of the MLEs lie to the left of or at the left red line, and 5\% lie to the right or at the right red line. The 90\% CI for $\vp$, rounded to the closest integer, is then somewhere around (1.4,7) when we use the percentile method.
\begin{figure}
    \centering
    %\includegraphics{}
    \begin{tikzpicture}[font=\small]
\begin{axis}[
ybar,
bar width=17pt,
xlabel={Value of $\hat{\eta}$, rounded to the closest integer},
ylabel={Number of $\hat{\eta}$},
ymin=0,
ytick={5,10,15,20,25,30,35,40},
xtick=data,
axis x line=bottom,
axis y line=left,
enlarge x limits=0.2,
]
\addplot[fill=myblue] coordinates {
    (0,2)
    (1, 6)
    (2, 14)
    (3, 29)
    (4, 36)
    (5, 36)
    (6, 15)
    (7, 7)
    (8, 4)
};
\end{axis}
\draw[dashed,red](1.6,0) -- (1.6,5.7);
\draw[dashed,red](5.27,0) -- (5.27,5.7);
\end{tikzpicture}
    \caption[Bootstrap Example]{Here we have plotted the MLEs of 150 bootstrap samples in a histogram. The red dashed lines represent the 5 and 95 percentiles.}
    \label{percentile_ci_example}
\end{figure}

