\newpage
\chapter{Background Theory}
%\section{Theory}
In this section we will go through some of the statistical theory used in this thesis. That includes the theorem of total probability, Bayes' theorem, the beta and gamma functions, Bayesian modelling, loss functions, the Softmax function, maximum likelihood estimation and bootstrapping.  

\section{The Theorem of Total Probability}
The theorem of total probability can be used when we have dependent variables and we wish to find the probability of one variable alone(?).
\begin{theorem}[Theorem of Total Probability, Continuous Variables]
If we have a continuous variable, $\Theta$, and a discrete variable, $U$, and both $P(U=u|\Theta=\theta)$ and  $f_\Theta(\theta)$ exist for all $\theta$, then we can find $P(U=u)$ using these \citep{schay2016introduction}. 
\begin{equation}
    \label{lawoftotprob}
    P(U=u) = \int_{-\infty}^{\infty} P(U=u|\Theta=\theta)f_{\Theta}(\Theta=\theta) \: \dd \theta.
\end{equation}
\end{theorem}

\begin{comment}
In \citet{schay2016introduction}, the theorem of total probability for continuous variables is stated as 
\begin{theorem}[Theorem of Total Probability, Continuous Versions]
 For a continuous random variable Y and any event A, if $f_{Y|A}$ and $f_Y$ exists for all y, then
\begin{equation}
   % \label{lawoftotprob}
    P(A) = \int_{-\infty}^{\infty}
    P(A|Y=y)f_Y(y) dy.
\end{equation}
\end{theorem}
\end{comment}




Consider, for example, two discrete random variables $U$ and $V$ that are conditionally independent given the continuous stochastic variable $\theta$. To find the probability that $U+V$ is equal to some integer $j$, we can use the theorem of total probability to condition on theta. Thus,
\begin{equation*}
    P(U+V=j) = \int_{-\infty}^\infty P(U+V=j|\Theta=\theta)f_{\Theta}(\Theta=\theta) \: d\theta
\end{equation*}
Later, we can exploit the conditional independence. If $\theta$ is a probability that is defined on the interval (0,1), this will be integrated on that interval, such that 
\begin{equation*}
    P(U+V=j) = \int_{0}^1 P(U+V=j|\Theta=\theta)f_{\Theta}(\Theta=\theta) \: d\theta.
\end{equation*}
Mer utfyllende her?




\section{Bayes' Rule}
Bayes' rule can be used to find conditional probabilities and distributions. 
\begin{theorem}[Bayes' Rule]
Consider two events, $A$ and $B$. We can find the probability of $A$ given event $B$ by the use of the probability of event $B$ given $A$ and the probabilities of the events $A$ and $B$ separately \citep{statinf}. Hence,
\begin{equation}
\label{bayesrule}
    P(A|B)=\frac{P(B|A)P(A)}{P(B)}.
\end{equation}
\end{theorem}




\begin{comment}
From \citet{schay2016introduction}, we have a version of Bayes' theorem for continuous variables which is stated as 
\begin{theorem}[Bayes' Theorem]
\label{bayestheorem}
For a continuous random variable Y and any event A with nonzero probability, if $P(A|Y=y)$ and $f_Y$ exist for all $y$, then
\begin{equation}
    \label{bayestheorem_eq}
    f_{Y|A}(y) = \frac{P(A|Y=y)f_Y(y)}{\int_{-\infty}^{\infty}P(A|Y=y) f_Y(y) dy}.
\end{equation}
Here $f_Y$is called the prior density of $Y$, and $f_{Y|A}$ its posterior density, referring to the fact that these are the densities of $Y$ before and after the observation of $A$. 
\end{theorem}

From \eqref{lawoftotprob} we see that the denominator is the probability of event $A$, \begin{equation*}
    \int_{-\infty}^{\infty}P(A|Y=y) f_Y(y) = P(A),
\end{equation*}
such that Bayes' theorem can be stated as
\begin{equation}
    \label{bayestheorem_eq2}
    f_{Y|A}(y) = \frac{P(A|Y=y)f_Y(y)}{P(A)}.
\end{equation}


Or:
\end{comment}

\begin{comment}
\begin{theorem}[Bayes' Theorem]
\label{bayestheorem}
Consider two continuous random variables, $U$ and $\Theta$, that both have a nonzero probability. If both $P(U=u|\Theta=\theta)$ and $P(\Theta=\theta)$ exist for all $\theta$, then 
\begin{equation}
    \label{bayesrule}
    P(\Theta=\theta|U=u) = \frac{P(U=u|\Theta=\theta)P(\Theta=\theta)}{\int_{-\infty}^{\infty} P(U=u|\Theta=\theta)P(\Theta=\theta) \: \dd \theta}.
\end{equation}
$P(\Theta=\theta)$ is the prior probability/density of $\Theta$, and it represents the prior knowledge we have about that parameter. $P(\Theta=\theta|U=u)$ is called the posterior of $\Theta$, which is the probability density after we have observed that $U=u$ \citep{schay2016introduction}. 
\end{theorem}


From \eqref{lawoftotprob} we see that the denominator is the probability that $U=u$,
\begin{equation*}
    \int_{-\infty}^{\infty} P(U=u|\Theta=\theta)P(\Theta=\theta) \: \dd \theta = P(U=u).
\end{equation*}
Thus, Bayes' theorem can be reformulated as
\begin{equation}
    \label{Bayesrule2}
     P(\Theta=\theta|U=u) = \frac{P(U=u|\Theta=\theta)P(\Theta=\theta)}{P(U=u)}.
\end{equation}


As an example, consider a random variable $U$ that is binomial distributed with parameters $n$ and $\theta$. Thus,
\begin{equation*}
    U|\theta \sim \mathrm{Binomial}(n,\theta).
\end{equation*}
Using Theorem \ref{bayestheorem} and \eqref{Bayesrule2}, we get that the posterior probability of $\Theta|U$ is
\begin{equation*}
    P(\Theta=\theta|U=u) = \frac{P(U=u|\Theta=\theta)P(\Theta=\theta)}{P(U=u)}.
\end{equation*}

\end{comment}

As an example, consider a discrete random variable, $U$. We can find the probability that $U$ is 7 and condition on it being different from 6 by using \eqref{bayesrule}. Thus
\begin{equation}
    P(U\geq 7|U\neq6) = \frac{P(U\neq6|U\geq7)P(U\geq7)}{P(U\neq6)}.
\end{equation}

\section{The Beta and Gamma Functions}
%The beta distribution is continuous on the interval between 0 and 1, and have two parameters \citep{statinf}. Consider a parameter $\theta$ is beta distributed with parameters $\gamma$ and $\kappa$, hence,
%\begin{equation*}
%    \theta \sim beta(\gamma,\kappa).
%\end{equation*}
%The probability density function of $\theta$ is then
% \begin{equation}
    % \label{betadistribution}
    % f(\theta|\gamma,\kappa) = \frac{1}{\mathrm{B}(\gamma,\kappa)} \theta^{\gamma-1}(1-\theta)^{\kappa-1},\:\: 0<\theta<1, \gamma>0, \kappa>0,
% \end{equation}
% where $\mathrm{B}(\gamma,\kappa)$ is the beta function, meaning that


Later, we will use the beta and gamma functions and some properties of these, thus, these are stated here. This theory can for example be found in \citet{statinf}. The gamma function for a parameter $\kappa$ is 
\begin{equation*}
    \label{gamma_func}
    \Gamma(\kappa) = \int_0^\infty t^{\kappa-1}e^{-t} \dd t.
\end{equation*}
A useful property of the gamma function is that it is recursive. Hence,
\begin{equation*}
    \Gamma(\kappa+1) = \kappa \Gamma(\kappa), \quad \kappa>0 .
\end{equation*}

Additionally, the beta function with parameters $\gamma$ and $\kappa$ is defined as
\begin{equation*}
    \mathrm{B}(\gamma,\kappa) = \int_0^1 \theta^{\gamma-1}(1-\theta)^{\kappa-1} \: \dd \theta.
\end{equation*}
We can express the beta function as a product of gamma functions. That yields
\begin{equation}
    \label{beta_as_gamma}
    \mathrm{B}(\gamma,\kappa) = \frac{\Gamma(\gamma)\Gamma(\kappa)}{\Gamma(\gamma+\kappa)}.
\end{equation}


% Using this and the fact that $\Gamma(1)=1$, we get the useful property that (not sure if I need this?)
% \begin{equation*}
    % \Gamma(n) = (n-1)!,
% \end{equation*}
% for all integers $n>0$.







%%%%%%%%%%%%%%%%%%%%%%%%%%%%%%%%%%%%%%%%%%%%%%%%%%%%%%%
\begin{comment}
\section{The binomial distribution}
Alternative title: The Bernoulli and binomial distributions

Assume we have a random variable, $X_j$, that is 1 with probability $\theta$ and 0 with probability $1-\theta$. $X_j$ is the Bernoulli distributed with parameter $\theta$, thus
\begin{equation*}
    X_j \sim \mathrm{Bernoulli}(\theta).
\end{equation*}
Consider a sample of $n$ draws from that Bernoulli distribution, $(x_1,x_2,...,x_n)$. If we define a stochastic variable, $Y$, as
\begin{equation*}
    Y = \sum_{j=1}^n X_j,
\end{equation*}
then $Y$ has a binomial distribution with parameters $n$ and $\theta$. Thus,
\begin{equation*}
    Y \sim \mathrm{Binomial}(n,\theta).
\end{equation*}
$Y$ is then the number of times we get 1 when drawing from the Bernoulli distribution $n$ times. The probability density function of $Y$ is
\begin{equation*}
    P(Y=y|n,\theta) = \binom{n}{\theta} \theta^y (1-\theta)^{n-y}, \quad y=0,1,2,...n.
\end{equation*}

example: the box task, probability that a box is red is $\theta$. $X_i=1$ when the box is red. then the probability that there are $y$ red boxes when all 12 boxes are opened is $\binom{12}{\theta} \theta^y (1-\theta)^{12-y}$.

\section{The beta distribution}
The beta distribution is continuous on the interval between 0 and 1, and have two parameters \citep{statinf}. Consider a parameter $\theta$ is beta distributed with parameters $\gamma$ and $\kappa$, hence,
\begin{equation*}
    \theta \sim beta(\gamma,\kappa).
\end{equation*}
The probability density function of $\theta$ is then
\begin{equation}
    %\label{betadistribution}
    f(\theta|\gamma,\kappa) = \frac{1}{\mathrm{B}(\gamma,\kappa)} \theta^{\gamma-1}(1-\theta)^{\kappa-1},\:\: 0<\theta<1, \gamma>0, \kappa>0,
\end{equation}
where $\mathrm{B}(\gamma,\kappa)$ is the beta function, meaning that
\begin{equation*}
    \mathrm{B}(\gamma,\kappa) = \int_0^1 \theta^{\gamma-1}(1-\theta)^{\kappa-1} \dd \theta.
\end{equation*}
We can express the beta function as a product of gamma functions, $\Gamma(\cdot)$. The gamma function for a variable $\gamma$ is
\begin{equation*}
    \label{gamma_func}
    \Gamma(\gamma) = \int_0^\infty t^{\gamma-1}e^{-t} \dd t.
\end{equation*}
A useful property of the gamma function is that it is recursive. Hence,
\begin{equation*}
    \Gamma(\gamma+1) = \gamma \Gamma(\gamma), \quad \gamma>0 .
\end{equation*}
Using this and the fact that $\Gamma(1)=1$, we get the useful property that (not sure if I need this?)
\begin{equation*}
    \Gamma(n) = (n-1)!,
\end{equation*}
for all integers $n>0$.


Expressing the beta function in terms of gamma functions yields
\begin{equation}
    \label{beta_as_gamma}
    \mathrm{B}(\gamma,\kappa) = \frac{\Gamma(\gamma)\Gamma(\kappa)}{\Gamma(\gamma+\kappa)}.
\end{equation}

Example of how the beta distribution is used. 

\end{comment}
%%%%%%%%%%%%%%%%%%%%%%%%%%%%%%%%%%%%%%%%%%%%%%%%%%%%%%%%




\section{Bayesian Modelling}
Consider a stochastic variable, $X$, that has a probability density function $f(x|\theta)$, where $\theta$ is a parameter upon which $X$ depends. In classical statistics, $\theta$ is said to be a fixed but unknown value. On the other hand, in Bayesian statistics, we consider $\theta$ as a stochastic variable. Thus, $\theta$ has a density function, $f(\theta)$. This is called the prior distribution as it represents the prior knowledge we have about $\theta$ before observing any data. That could be our own subjective believes about the parameter, or other previously collected data or studies. One could also choose a prior distribution that does not say anything about the parameter at all. This is called a non-informative prior, and it is often used when we have none or little prior information about the parameter \citep{givens2012computational}. 
If we have collected data, we can update our prior beliefs with the information we get from that data. The resulting distribution is called the posterior distribution of $\theta$. This can be found using Bayes' theorem, and it includes both the prior information we have and the new information we get from the data. 

Consider a stochastic variable, $u$, that has a sampling distribution $f(u|\theta)$, and let $f(u)$ be the marginal distribution of $u$. Additionally, let $f(\theta)$ be the prior distribution of $\theta$, hence our prior beliefs of the parameter. Using Bayes' rule as it is stated in \eqref{bayesrule}, we get that the posterior distribution of $\theta$ given $u$, $f(\theta|u)$, can be expressed as \citep{statinf}
\begin{equation*}
    f(\theta|u) = \frac{f(u|\theta)f(\theta)}{f(u)}.
\end{equation*}
We can sometimes exploit the fact that the posterior distribution is proportional to the numerator in the above expression. This is because the denominator is a normalising constant. Then,
\begin{equation}
    \label{posterior_proportional}
    f(\theta|u) \propto f(u|\theta)f(\theta).
\end{equation}
If this have the form of a known distribution, that is the posterior distribution. 

As an example, consider a random variable, $U$, that is binomial distributed with parameters 12 and $\theta$. Thus,
\begin{equation*}
    U|\Theta=\theta \sim \mathrm{Binomial}(12,\theta).
\end{equation*}
Thus, the probability that we have $u$ successes out of twelve, given $\theta$, is
\begin{equation*}
    f(u|\theta) = \binom{12}{u} \theta^{u} (1-\theta)^{12-u}.
\end{equation*}
We choose a beta prior for $\theta$ with parameters $\gamma$ and $\kappa$ (burde jeg si noe om hvorfor vi velger en beta prior her, eller heller gjøre det lenger ned?). Hence,
\begin{equation*}
    \Theta \sim \mathrm{Beta}(\gamma,\kappa). 
\end{equation*}
The prior density of $\Theta$ is then
\begin{equation}
    \label{betadistribution}
    f(\theta) = \frac{1}{\mathrm{B}(\gamma,\kappa)}\theta^{\gamma-1}(1-\theta)^{\kappa-1}.
\end{equation}
We can find the posterior distribution using \eqref{posterior_proportional}. Thus,
\begin{equation*}
    \begin{aligned}
        f(\theta|u) 
        &\propto f(u|\theta)f(\theta)\\[6pt]
        &\propto \binom{12}{u} \theta^{u} (1-\theta)^{12-u} \frac{1}{\mathrm{B}(\gamma,\kappa)}\theta^{\gamma-1}(1-\theta)^{\kappa-1}\\[6pt]
        &\propto C \: \theta^{u+\gamma-1}(1-\theta)^{12-u+\kappa-1},
    \end{aligned}
\end{equation*}
where $C$ is a constant. We can see that this is proportional to a beta distribution like the one in \eqref{betadistribution}, but in this case with parameters $u+\gamma$ and $12-u+\kappa$. Hence, the posterior distribution is a beta distribution with these parameters, 
\begin{equation*}
    \Theta|U=u \sim \mathrm{Beta}(u+\gamma,12-u+\kappa).
\end{equation*}






\section{Loss Functions}
A loss function typically says something about the cost, or loss, of an action related to a parameter. 
Let $\Delta$ be the action space, consisting of all the actions that can be done and $\Theta$ be the parameter space, where $\theta$ is the parameter you make decisions regarding. If the decision you make is far away from the parameter, the cost of making that decision will be high, and then we say that the loss will be big. A decision close to the parameter will give a small loss \citep{statinf}. The loss function gives the loss of making a decision. 

A commonly used loss function is the absolute error loss function. Let $\delta$ be the 
action represented as a real number, and let $\theta$ be the parameter we make decisions regarding. The absolute error loss function is then defined as
\begin{equation*}
    L(\delta,\theta) = |\delta-\theta|.
\end{equation*}
The loss is zero if $\delta=\theta$. A special case of this loss function is the 0-1-loss. Then the loss is either 0 or 1, and only two decisions can be made. Then, 
\begin{equation*}
    L(\delta,\theta) = I(\delta \neq \theta),
\end{equation*}
where $I$ is an indicator function such that
\begin{equation*}
    L(\delta,\theta) =
    \begin{cases}
        0,&  \text{if } \delta = \theta, \\
        1,&  \text{if } \delta \neq \theta.
    \end{cases}
\end{equation*}
In some cases, we would like to find the expected value of the loss function. Taking the expected value of an indicator function gives the probability that the event inside the function is happening \citep{algdat}. Hence, taking the expectation of the loss function above gives
\begin{equation}
\label{expectation_of_loss_func_general}
    E[L(\delta,\theta)] = E[I(\delta\neq\theta)] = P(\delta\neq\theta).
\end{equation}



As an example, consider the box task with twelve boxes  that could be either blue or red once they are opened. Let $Z$ be the colour that is in majority when all twelve boxes are opened, the true majority colour. We define it such that
\begin{equation*}
    Z = 
    \begin{cases}
        0,& \text{if blue is the true majority colour,} \\
        1,& \text{if red is the true majority colour.}
    \end{cases}
\end{equation*}
Also, let $\delta$ be the colour that the participant chooses to be the majority colour. In the same manner as $Z$, $\delta$ is 0 if the participant chooses blue as the majority colour and 1 if the participant chooses red. We can then define a loss function for the choice that the participant makes. This is a 0-1 loss, and the loss function can therefore be
\begin{equation*}
    L(\delta,Z) = I(\delta \neq Z),
\end{equation*}
Then, the loss is 0 if the participant chooses the right colour as the majority colour, and 1 if the wrong colour is chosen. 

To find the expected loss, we take the expectation of the loss function. As $Z$ depends on the colour of the twelve boxes, so does the expected loss. We define a stochastic variable, $X_i$, that represents the colour of the $i$-th opened box, such that 
\begin{equation*}
    X_i =
    \begin{cases}
        0,& \text{if box }i \text{ is blue,}\\
        1,& \text{if box }i \text{ is red.}
    \end{cases}
\end{equation*}
Let $\textbf{X}$ denote the colours of the twelve boxes, such that
\begin{equation*}
    \textbf{X} = (X_1,X_2,...,X_{12}).
\end{equation*}
When we find the expected loss of choosing either blue or red as dominant colours, we condition on the true colours of the boxes. Thus, the expectation of the loss function is
\begin{equation*}
    E[L(\delta,Z)|\textbf{X}] = E[I(\delta \neq Z)|\textbf{X}].
\end{equation*}
As in \eqref{expectation_of_loss_func_general}, this expectation is the probability that $\delta \neq Z$. Thus,
\begin{equation*}
    E[L(\delta,Z)|\textbf{X}] = P(\delta \neq Z|\textbf{X}).
\end{equation*}



%(Not sure if I should include this as this was fine when the loss func was if red or blue was teh right color, but now I have 3 decisions, and not two, and then teh 0-1-loss does not count, but it counts if we only talk about the two first decisions, decision 0 and 1.)



\begin{comment}
Absolute error loss. Special case of that is the 0-1-loss. that can be expressed as an indicator function. And then the expectation of the indicator function. 


Maybe the loss functions for choosing blue and red should be
\begin{equation*}
    L(\delta =0) = I(d=0)
\end{equation*}
and 
\begin{equation*}
    L(\delta =1) = I(d=1)
\end{equation*}
and for unlimited choosing to open the next box
\begin{equation*}
    L_i(\delta_i=2,d_i)=\alpha+L_{i+1}(\delta_{i+1},d_{i+1})?
\end{equation*}

$\Delta$ as the action space and $\delta$ as the parameter?

hva er en loss function? says something about the cost of an action tied/in relation to(?) to a parameter. Typical input is the action and a parameter. 
Typisk eksempel? 0-1-loss? and indicator variables

example. 
\end{comment}


\section{The Softmax Function} \label{section_theory_softmax}
The softmax function is commonly used in classification problems with more that two classes \citep{softmax}. Let there be $K$ classes, and let $\delta_i$ represent each one of them with $i \in (0,1,2,...,K-1)$. Additionally, let $\EE_{\delta_i}$ be functions depending on the classes, and $\textbf{x}$ be data. Then, the probability density function for each class could be found using a softmax function. Having some parameter $\eta$, this could be
\begin{equation*}
    \begin{aligned}
        f(\delta_i|\textbf{x}) = \frac{\text{exp}(- \eta \EE_{\delta_i})}{\sum_{j=0}^{K-1} \text{exp}(-\eta \EE_{\delta_j})}.
    \end{aligned}
\end{equation*}
Er et riktig å ha $|\textbf{x}$ her? jeg vil jo ha $|\alpha,\beta,\eta$ i min modell? burfe jeg ha $|\eta$, også senere introdusere $\alpha$ og $\beta$. slik at det heller blir som under?
\begin{equation*}
    \begin{aligned}
        f(\delta_i|\eta) = \frac{\text{exp}(- \eta \EE_{\delta_i})}{\sum_{j=0}^{K-1} \text{exp}(-\eta \EE_{\delta_j})}.
    \end{aligned}
\end{equation*}

These classes could for example be the three choice you have each time a box is opened in the box task. These choices are that blue is the majority colour, or that red is, denoted $\delta=0$ and $\delta=1$, respectively. The last choice is to open another box, which is denoted $\delta=2$. Then, $\EE_0$ could be the expected loss when choosing that blue is the majority colour, in the same manner as in (reference to equation here). Additionally, $\EE_1$ and $\EE_2$ might be the expected loss of choosing red as the majority colour and of opening another box, respectively. If these expected losses depend on some parameters $\alpha$ and $\beta$, the probability mass functions would be
\begin{equation}
\label{softmax_1}
    f(\delta|\alpha,\beta,\eta) = \frac{\text{exp}(- \eta \EE_{\delta}(\alpha,\beta))}{\sum_{j=0}^{2} \text{exp}(-\eta \EE_{j}(\alpha,\beta))}.
\end{equation}

\section{Maximum Likelihood Estimation}
\label{section_theory_mle}

Maximum likelihood estimation is used to find estimates for parameters in a distribution (also called estimators?). These are the estimates that, as the name implies, maximises the likelihood, and for short, we call them MLE's. Consider $n$ samples, $\Delta_1,\Delta_2,...,\Delta_n$, from a population that has probability mass function $f(\delta_i|\theta_1,\theta_2,...,\theta_k)$, where the $\theta$'s are the parameters in the probability mass function. Then the likelihood function is defined as
\begin{equation*}
    L(\theta_1,\theta_2,...,\theta_k|\delta_1,\delta_2,...,\delta_n) =  \prod_{i=1}^{n} f(\delta_i|\theta_1,\theta_2,...,\theta_k).
\end{equation*} 
The MLE's are then the estimates that maximises this function, and they are often denoted $\hat{\theta}_i$. It is often hard to maximize the likelihood function, then it might be easier to take the logarithm of the likelihood function and maximize that instead. This is called the log likelihood function, and is normally denoted as $l$. Thus,
\begin{equation*}
    \begin{aligned}
        l(\theta_1,\theta_2,...,\theta_k|\delta_1,\delta_2,...,\delta_n) 
        =& \text{log}\big(L(\theta_1,\theta_2,...,\theta_k|\delta_1,\delta_2,...,\delta_n)\big)\\
        =& \text{log}\big(\prod_{i=1}^{n} f(\delta_i|\theta_1,\theta_2,...,\theta_k) \big).
    \end{aligned}
\end{equation*}
As the logarithm of products is the sum of the logarithms, we get that the log likelihood is
\begin{equation*}
    \begin{aligned}
        l(\theta_1,\theta_2,...,\theta_k|\delta_1,\delta_2,...,\delta_n) = \sum_{i=1}^n \text{log}(f(\delta_i|\theta_1,\theta_2,...,\theta_k)).
    \end{aligned}
\end{equation*}
Maximizing this will give the same maximum point as if we maximize the likelihood function. 

As an example, consider that the $\delta_i$'s have probability mass function as in \eqref{softmax_1}. The parameters that we want to find estimates for are then $\alpha$, $\beta$ and $\eta$. If we have $n$ samples of $\delta_i$, the likelihood function would be
\begin{equation*}
    \begin{aligned}
        L(\alpha,\beta,\eta|\delta_1,\delta_2,...,\delta_n) 
        =& \prod_{i=1}^{n} f(\delta_i|\alpha,\beta,\eta)\\[6pt]
        =& \prod_{i=1}^{n} \frac{\text{exp}(- \eta \EE_{\delta_i}(\alpha,\beta))}{\sum_{j=0}^{K-1} \text{exp}(-\eta \EE_{\delta_j}(\alpha,\beta))}.
    \end{aligned}
\end{equation*}
The log likelihood would then be
\begin{equation*}
    \begin{aligned}
        l(\alpha,\beta,\eta|\delta_1,\delta_2,...,\delta_n) =& \sum_{i=0}^N \text{log}\Big( \frac{e^{-\eta \EE_{\delta_{i}(\alpha,\beta)}}}
        {\sum_{j=0}^{K-1} \text{exp}(-\eta \EE_{\delta_j}(\alpha,\beta))}\Big) \\[6pt]
        =& \sum_{i=0}^N -\eta \EE_{\delta_{i}} 
        - \text{log}\Big(\sum_{j=0}^{K-1} \text{exp}(-\eta \EE_{\delta_j}(\alpha,\beta))\Big).
    \end{aligned}
\end{equation*}
The maximum likelihood estimators of $\alpha$, $\beta$ and $\eta$ would then be the values that maximises this log likelihood function, frequently denoted as $\hat{\alpha}$, $\hat{\beta}$ and $\hat{\eta}$.




\section{Bootstrapping}
\label{section_theory_bootstrap}
Bootstrapping is a way to draw or simulate many samples from one single dataset. If you have a dataset, you can draw random samples from them with replacement, to construct bootstrap samples \citep{bootstrap}. 
If you for example have data $\textbf{x}=(x_1,x_2,x_3,x_4,x_{5})$, then a bootstrap sample might be $(x_5,x_5,x_2,x_3,x_1)$ and another might be $(x_2,x_{4},x_{2},x_{2},x_{1})$. These are then resampled versions of $\textbf{x}$. Thus, the bootstrap samples consists of elements from the original dataset, but some of them might not appear at all in a bootstrap sample while others might appear twice. This is called nonparametric bootstrapping. If we, for example, have found the maximum likelihood estimate (MLE) of a parameter, $\eta$, we can use these bootstrap samples to for example find the standard error or confidence interval (CI) for $\eta$.

If we have a distribution for the $x$'s, we could instead of using a nonparametric bootstrapping, use a parametric version. The we simulate new $x$'s based on the MLE of $\eta$. If the distribution is $f(x|\eta)$, and the MLE of $\eta$ is denoted $\hat{\eta}$, then we simulate new $x$'s with $f(x|\hat{\eta})$. As for the nonparametric bootstrap, we can for example find standard errors and confidence intervals (CIs) for parameters. 

\subsection{Confidence Intervals with Bootstrap Samples}
When we have the bootstrap samples, there are multiple methods for finding CIs. One method is the percentile method. Consider a situation with $B$ bootstrap samples. If the intention is to find a CI for the MLE of $\eta$, $\hat{\eta}$, we find the MLE of each of the $B$ samples and plot those samples in a histogram. If we want to find a 90\% confidence interval, then we find the 5-th and 95-th percentiles. The 5-th percentile is then the value of $\hat{\eta}$ in the histogram where 5\% of the samples are below. The 95-th is where 5\% of the values are above. This is visualised in Figure \ref{percentile_ci_example}. Here we have 150 bootstrap samples, and the MLE of $\eta$ is found for each sample. These values are plotted in a histogram, where the values of the MLEs is rounded to the closest integer. The red dashed lines represents the 5 and 95 percentiles. That means that 5\% of the MLEs lie to the left of or at the left red line, and 5\% lie to the right or at the right red line. This is then the 90\% confidence interval of $\eta$ when we use the percentile method.
\begin{figure}
    \centering
    %\includegraphics{}
    \begin{tikzpicture}[font=\small]
\begin{axis}[
ybar,
bar width=17pt,
xlabel={Value of $\hat{\eta}$, rounded to the closest integer},
ylabel={Number of $\hat{\eta}$},
ymin=0,
ytick={5,10,15,20,25,30,35,40},
xtick=data,
axis x line=bottom,
axis y line=left,
enlarge x limits=0.2,
]
\addplot[fill=myblue] coordinates {
    (0,2)
    (1, 6)
    (2, 14)
    (3, 29)
    (4, 36)
    (5, 36)
    (6, 15)
    (7, 7)
    (8, 4)
};
\end{axis}
\draw[dashed,red](1.6,0) -- (1.6,5.7);
\draw[dashed,red](5.27,0) -- (5.27,5.7);
\end{tikzpicture}
    \caption{Maximum likelihood estimate of 150 bootstrap samples, rounded to the closest integer. The red dashed lines represent the 5 and 95 percentiles.}
    \label{percentile_ci_example}
\end{figure}

The percentile method is simple to both understand and implement. However, these confidence intervals might be biased. Then, one could instead use approaches such as \textit{bias corrected and accelerated} intervals or \textit{approximate bootstrap confidence} intervals. For simplicity, we are in this report using the percentile confidence intervals. 
