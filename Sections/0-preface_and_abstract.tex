
\section*{Preface}
\addcontentsline{toc}{chapter}{Preface}
This is the TMA4900 - Industrial Mathematics, Master's Thesis, which is part of my Master of Science in Applied Physics and Mathematics with an Industrial Mathematics major. I wish to thank my supervisor, Håkon Tjelmeland, for following up on me, answering my questions, and for helping me all the way through. I would also like to thank Kristoffer Klevjer and Gerit Pfuhl for introducing me to this project and providing me with data. 

\cleardoublepage
%\afterpage{\null\newpage}

\clearpage
\begin{abstract}
\thispagestyle{plain}
\setcounter{page}{3}
\addcontentsline{toc}{chapter}{Abstract}
    Delusions are one of the main symptoms of schizophrenia, and delusion prone individuals have been linked to a 'jumping to conclusions' bias. That means drawing conclusions without having sufficient information. An information sampling task called the box task has been proposed to find if participants have this bias. In the box task we have a grid of grey boxes that, when they are opened, display one out of two colours. The participants are told that one of the colours always is in majority, and their task is to find out which one. Two versions of the box task are used here, one where the participants can open as many boxes as they want and another where the test terminates when a random box is opened. These are called the unlimited and limited versions, respectively. 
    In this report, we find an Ideal Observer solution of the box task, where an Ideal Observer is someone who would make the optimal choice each time a box is opened.
    We have data from 76 participants that have done both versions of the box task, and in this report, we model how they make decisions. We find parameter estimates using maximum likelihood and find confidence intervals using parametric bootstrapping. 
    
    We have three parameters in the model. $\alpha$ is a small loss or penalty we get each time a box is opened, $\beta$ is the loss we get if the test terminates in a limited trial and $\eta$ is a measure of how good the decisions the participant makes are. We find that in the limited version, most participants have $\alpha=0$, which might indicate that this parameter is unnecessary to include here. We also find that an Ideal Observer not necessarily makes the best decisions in the trials the participants have done here as the test might often terminates before an Ideal Observer would choose majority colour.
    Ta med noe om sensitivity. 
\end{abstract}
\cleardoublepage
%\afterpage{\null\newpage}

\clearpage
\selectlanguage{norsk}
\begin{abstract}
\thispagestyle{plain}
\setcounter{page}{5}
\addcontentsline{toc}{chapter}{Sammendrag}
    Skriv norsk sammendrag her
\end{abstract}
\cleardoublepage
%\afterpage{\null\newpage}
\selectlanguage{english}