
\section*{Preface}
\addcontentsline{toc}{chapter}{Preface}
This is the TMA4900 - Industrial Mathematics, Master's Thesis, which is part of my Master of Science in Applied Physics and Mathematics with an Industrial Mathematics major. I wish to thank my supervisor, Håkon Tjelmeland, for following up on me, answering my questions, and for helping me all the way through. I would also like to thank Kristoffer Klevjer and Gerit Pfuhl for introducing me to this project and providing me with data. 

\cleardoublepage
%\afterpage{\null\newpage}

\clearpage
\begin{abstract}
\thispagestyle{plain}
\setcounter{page}{3}
\addcontentsline{toc}{chapter}{Abstract}
    Delusions are one of the main symptoms of schizophrenia, and delusion prone individuals have been linked to a 'jumping to conclusions' bias. That means drawing conclusions without having sufficient information. An information sampling task called the box task has been proposed to find if participants have this bias. In the box task we have a grid of grey boxes that, when they are opened, display either the colour red or the colour blue. The participants are told that one of the colours always is in majority, and their task is to find out which one. Two versions of the box task are used here, one where the participants can open as many boxes as they want and another where the test terminates when a random box is opened. These are called the unlimited and limited versions, respectively. 
    In this report, we find an Ideal Observer solution of the box task, where an Ideal Observer is someone who would make the optimal choice each time a box is opened.
    We have data from 76 participants that have done both versions of the box task, and in this report, we model how they make decisions. We have three parameters in the model. The parameter $\alpha$ is a small loss or penalty we get each time a box is opened, $\beta$ is the loss we get if the test terminates in a limited trial and $\eta$ is a measure of how good the decisions the participant make are.
    We estimate these parameters for each participant using maximum likelihood estimation and find confidence intervals using parametric bootstrapping and look at the sensitivity to the hyperparameters in the prior distribution for $\Theta$, which is the probability that a box is red.
    
    We find that this model is a good fit for the participants that make good choices, but not for the ones that make bad choices. Parametric bootstrapping makes the confidence intervals for the participants that make optimal choices be zero, meaning that, for these participants, this is not the best choice of method for finding these intervals. The parameters in the unlimited version are not sensitive to changes in the hyperparameters, whereas in the limited version, the estimates tend to be smaller for smaller values of the hyperparameters. 
    %We find that in the limited version, most participants have $\alpha=0$, which might indicate that this parameter is unnecessary to include here. We also find that an Ideal Observer not necessarily makes the best decisions in the trials the participants have done here as the test might often terminates before an Ideal Observer would choose majority colour.
    %Ta med noe om sensitivity. 
\end{abstract}
\cleardoublepage
%\afterpage{\null\newpage}

\clearpage
\selectlanguage{norsk}
\begin{abstract}
\thispagestyle{plain}
\setcounter{page}{5}
\addcontentsline{toc}{chapter}{Sammendrag}
    Et av de viktigste symptomene på schizofreni er vrangforestillinger. Mange med dette symptomet har vist seg å trekke slutninger uten å ha nok informasjon, det vil si at de har et 'jumping to conclusions' (JTC) bias. Bokstesten (the box task) har blitt foreslått å bruke for å finne ut om noen har et JTC bias. Da ser man tolv grå bokser i et rutenett. Man åpner en og en boks, og bak hver boks skjuler det seg en av to farger, rød eller blå. Deltakerne får beskjed om at en av fargene alltid er i majoritet og at de skal finne ut hvilken det er. Vi bruker to versjoner av bokstesten, en hvor man får åpne alle de tolv boksene og en annen hvor deltakerne får beskjed om at testen terminerer når en tilfeldig boks åpnes. Disse kalles henholdsvis ubegrenset og begrenset versjon. 
    I denne rapporten finner vi en Ideell Observatør løsning, hvor en Ideell Observatør er en deltaker som alltid tar optimale valg. Vi har data fra 76 personer som har gjort begge versjoner av bokstesten, og vi modellerer hvordan disse tar valg når de tar testen. Vi har tre parametere. Parameteren $\alpha$ representerer et lite tap man får hver gang en boks åpnes og $\beta$ er det tapet man får hvis testen terminerer i den begrensete versjonen. Den siste parameteren, $\eta$, sier noe om hvor gode valg man tar. Vi estimerer disse parameterne med sannsynlighetsmaksimering og finner konfidensintervaller ved hjelp av parametrisk bootstrapping. Deretter ser vi på hvor sensitive resultatene er når vi forandrer på hyperparameterne i apriorifordelingen til $\Theta$ som er sannsynligheten for at en boks er rød. 
    
    Vi finner at modellen passer bra hvis deltakerne tar gunstige valg, men ikke fullt så bra hvis de tar dårlige valg. Lengden på konfidensintervallene til individene som tar optimale eller nesten optimale valg blir null. For disse deltakerne er derfor ikke parametrisk bootstrapping den beste måten å finne disse intervallene på. Resultatene i den ubegrensede versjonen påvirkes lite når vi forandre på hyperparameterene, i motsetning til den begrensede versjonen hvor mange deltakere får lavere estimater av parameterne.  
\end{abstract}
\cleardoublepage
%\afterpage{\null\newpage}
\selectlanguage{english}