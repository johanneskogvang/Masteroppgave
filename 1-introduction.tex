\newpage
\section{Introduction}
Schizophrenia is a psychotic disorder where at least two of the symptoms delusions, hallucinations, disorganized speech, grossly disorganized or catatonic behaviour or negative symptoms such as reduced emotional expressions and lowered motivation, have to be present. 
Delusions are beliefs that will not change if contradicting evidence is presented. The most common type of delusions are persecutory delusions. The person having those kinds of delusions thinks that they will be hurt, injured, tormented or so on by others. Referential delusions are also common. Then one are putting meaning into comments, gestures and actions, thinking that they are about oneself, when they not necessarily are. Completely improbable beliefs are called bizarre delusions. These are delusions others find far-fetched, and they are things that cannot happen in real life. A bizarre delusion could for example be that a person believes that their organs have been removed replaced by someone else's organs without there being any scars or other evidence of that happening. A delusion that is not bizarre could be that you thinks you are under police surveillance without there being any evidence of that. It might be hard to distinguish between delusions and strongly held ideas. The main distinction is about the degree of conviction, and how much or little the beliefs can be amended when contradicting facts are presented. 
Hallucinations are sensory impressions that happen without any external stimulus. Hence, others usually does not experience the same impressions. The hallucinations are perceived as normal experiences to the person having them. For schizophrenic people, they often occur as voices which can be distinguished from the persons own thoughts. 
Disorganized speech means that one are switching between topics rapidly and giving unrelated answers to questions. 
Grossly disorganized behavior is also referred to as abnormal behavior, and catatonic behavior means that one has a dampened reaction to things happening around oneself. (All of the above is from DSM-5).

Delusions are one of the main features/characteristics of schizophrenia as it appears with about three out of four of those diagnosed \citep{garety2011}. Researchers have been trying to understand how the delusions are formed and maintained in order to improve treatment \citep{dudley_meta_2016}. One important finding is that deluded individuals seem to make decisions based on less evidence than both healthy and psychiatric controls. This is often referred to as a "jumping to conclusions" (JTC) bias \citep{dudley_meta_2016}. A person with this bias might reach decisions or form beliefs before ending up at realistic ideas/conclusions. Hence, they might be more prone to accept unrealistic ideas, and thus they are more prone to delusions \citep{dudley_meta_2016}. The hope is that if we can detect the JTC bias, we can reduce the delusional thinking, and thus prevent delusions \citep{dudley_meta_2016}.

The JTC bias is traditionally tested with a probabilistic reasoning task called the beads task. The participants are presented with two jars of beads of two colours, for example red and blue. The two jars have opposite ratios of each colour, meaning that if the first have 85\% red beads and 15\% blue, the second has 15\% red and 85\% blue beads. The participants are told that beads are drawn from one of the jars, and their task is to find out which one that is. They are told to choose only when they are completely sure, and they draw as many beads as they want. The beads are drawn sequentially, and after each draw the participants are asked if they want to choose which jar beads are drawn from or if they want to draw more beads. One are usually said to have a JTC bias if one decide after one or two beads \citep{moritz2017}. However, the beads task has shown to pose some problems.

Some of the first to use the beads task were \citet{huq1988}. Already then they presented some of the problems with using the beads task. They used an 85-15 ratio of the beads. When the two first beads that are drawn are of the same colour, it is a 97\% probability that the beads are from the jar with 85\% of he beads in that colour. One might therefore argue that choosing jar at that point is reasonable, and that it does not show a JTC bias. Deluded individuals make decisions earlier that the control groups, but \citeauthor{huq1988} argue that non-deluded individuals are more conservative, and that people with delusions simply cancel out that bias when gathering less information. In an article by \citet{moritz2017}, other problems with the beads task are discussed. For example that many participants seem not to grasp(?) that all the beads are drawn form the same jar. This makes them think that after each bead is drawn, they have to guess which jar that bead is drawn from. These participants are then classified to have a JTC bias. We can also see that it is common to make logical errors due to miscomprehension. In an article by \citet{moritz2005}, they found that 52\% of the schizophrenic participants and 23\% of the healthy controls had at least one response that was not logical. The participants that misunderstand are more likely to choose early. \citeauthor{moritz2017} further states that the beads task is correlated with intelligence. Lack of intelligence might be a reason for or a confound for misunderstanding the task. It is also stated that confidence influences decision-making. The participants are asked to choose when they are completely sure which jar the beads are drawn from, resulting in more confident participants to decide earlier. Therefore, we might think that hasty decisions might be because the participants like to take risks or that they are not cautious, or both. However, other tasks that accounts for confidence also display a JTC bias with the delusion-prone participants. Additionally, there is only a one-dimensional sequence of events in the beads task. Thus, it is harder to find different versions to test multiple times. 

The box task has been suggested as an alternative to the beads task. Here, the participants are presented with a grid of a set number of boxes. When a box is opened, one of two colours is displayed. They are told that one of the colours are always in majority, and their task is to find out which one \citep{moritz2017}. In the version of the box task used here, there are twelve boxes. The two colours vary in the different trials as the participants might prefer one colour over other. However, for simplicity, we are operating with red and blue boxes in this report. We also use two different versions of the box task. The first one is an unlimited one, where the participants can open as many boxes as they want, even until all twelve boxes are opened, before reaching a decision. In the second version, the participants are told that the test will terminate at a random point. If the test terminates before the participant has decided what the majority colour is, this counts as a failed trial. We call this the limited version. (something about: here they are not said to answer when they are completely sure? compared to the beads task, where they are said to answer when comp sure)

In this report we are firstly going to find an Ideal Observer solution of the box task. 
Assuming beta binomial etc. 


-we are to find an ideal observer solution. assuming beta-binomial.
(should i say something about the fact that i have wound another ideal observer solution assuming other things in the specialization thesis?)




-later we will model the decisions that participants do using data from people that have taken the test. Using both the ideal observer solution used here, and another found in the specialization thesis, assuming uniform. 
